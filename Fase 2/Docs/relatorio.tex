
\documentclass[12pt]{article}
\usepackage{lmodern}
\usepackage[T1]{fontenc}
\usepackage[portuges]{babel}
\usepackage[utf8]{inputenc}
\usepackage{a4}
\usepackage{textgreek}
\usepackage{epstopdf}
\usepackage{graphicx}
\usepackage{fancyvrb}
\usepackage{amsmath}
\usepackage{float}
\usepackage{listings}
%\renewcommand{\baselinestretch}{1.5}


\begin{document}

\begin{titlepage}

\newcommand{\HRule}{\rule{\linewidth}{0.5mm}} % Defines a new command for the horizontal lines, change thickness here

\center % Center everything on the page
    
%----------------------------------------------------------------------------------------
%	HEADING SECTIONS
%----------------------------------------------------------------------------------------

\textsc{\LARGE Universidade do Minho}\\[1.5cm] 
\textsc{\Large Mestrado Integrado em Engenharia Informática}\\[0.5cm] 
\textsc{\large Computação Gráfica}\\[0.5cm]

%----------------------------------------------------------------------------------------
%	TITLE SECTION
%----------------------------------------------------------------------------------------

\HRule \\[0.4cm]
{ \huge \bfseries Transformações Geométricas}\\[0.4cm] 
\HRule \\[1.5cm]
    
%----------------------------------------------------------------------------------------
%	AUTHOR SECTION
%----------------------------------------------------------------------------------------

\begin{minipage}{0.4\textwidth}
\begin{flushleft} \large
\emph{Grupo:}\\
Etienne Costa A76089 \\
Joana Cruz A76270 \\
Rafael Alves A72629 \\
Maurício Salgado A71407 \\
\end{flushleft}
\end{minipage}
~
\begin{minipage}{0.4\textwidth}
\begin{flushright} \large
\emph{Docente:} \\
António Ramires\\
\end{flushright}
\end{minipage}\\[2cm]

%----------------------------------------------------------------------------------------
%	DATE SECTION
%----------------------------------------------------------------------------------------

{\large \today}\\[2cm]

%----------------------------------------------------------------------------------------
%	LOGO SECTION
%----------------------------------------------------------------------------------------

\includegraphics[scale=0.3]{uminho}\\
    
%----------------------------------------------------------------------------------------

\vfill % Fill the rest of the page with whitespace

\end{titlepage}
\tableofcontents
\newpage
\section{Introdução}
O relatório apresentado diz respeito à segunda fase  do projeto proposto no âmbito da unidade curricular de
Computação Gráfica. O trabalho consiste na criação de um cenário através do parsing de ficheiros XML e aplicação de várias
transformações geométricas hierquicamenjd em OpenGL tal como translações, rotações e escalas.

\section{Leitura e Processamento de um ficheiro XML}
O processamento de um cenário em formato XML pode ser visto como duas fases
distintas:
\begin{itemize}
\item Leitura e Parsing do cenário – Consiste na abertura do ficheiro que contém o
cenário que em modo de leitura e extração da respetiva hierarquia em XML. Nesta
fase são também retiradas as componentes que caraterizam uma
transformação geométrica (translação, escala ou rotação) ou desenho de uma
primitiva.
\item Armazenamento nas estruturas de dados – De modo a se conseguir redesenhar
um modelo quantas vezes for necessário, as instruções que caraterizam uma
transformação geométrica são armazenadas em estruturas adequadas às
mesmas.
\end{itemize}
Para uma melhor organização do código, optou-se por separar estas duas fases do
programa Engine num conjunto de módulos apropriados às mesmas. As duas subsecções
seguintes descrevem com detalhe a organização desses módulos. 
\section{Armazenamento de dados em classes}
\subsection{Vertex}
Esta classe representa um ponto num referencial a três dimensões, com as
coordenadas x, y e z, o que se torna bastante étil para a representação dos
vértices utilizados posteriormente para o desenho dos triângulos que elaboram as figuras primitivas.
\begin{lstlisting}
class Vertex{
	public:
		float x;
		float y;
		float z;
		Vertex();
		Vertex(float xx, float yy, float zz);
		~Vertex();
};
\end{lstlisting}
\subsection{Model}
Armazena todos os pontos para a criação das figuras primitivas (plane, box,
cone, sphere e torus) estão presentes e descritos neste ficheiro, assim como a cor escolhida para o modelo.
\begin{lstlisting}
class Model{
	public:
		string fileName;
		vector<Vertex> vertexes;
		Vertex color;
		Model();
		Model(string path, Vertex color);
		~Model();
};
\end{lstlisting}
\subsection{Group}
Tal como sugere o nome, esta é a classe principal de armazenamento que
guarda os dados retirados do ficheiro XML.
Nesta estrutura é possével armazenar todas as informaçõoes que estão associadas a uma determinada figura tal 
como as transformações. A figura, por sua vez, é representada por um ficheiro com extensão
3d onde se pode encontrar todos os pontos que a constitui. Sendo assim,
achou-se conveniente armazenar igualmente o nome do ficheiro e todos os
vértices contidos neste.
\begin{lstlisting}
class Group{
	public:
		Vertex rotation;
		float rotationAngle;
		Vertex translation;
		Vertex scale;
		vector<Model> models;
		vector<Group> subGroups;
		Group(void);
		Group(Vertex rotation, float rotAngle, Vertex translation, 
		Vertex scale, vector<Model> models, vector<Group> subGroups);
		~Group();
};
\end{lstlisting}
\section{Parser}
Esta classe é fundamento para o bom funcionamento do engine porque é
este que efetua o parsing do ficheiro XML. Desta forma, o Parser é responsável
por inserir toda a informa¸c˜ao encontrada no documento XML num vetor
de Struct.
\begin{itemize}
\item ParseXMLFile - sendo o cenário sempre um conjunto de grupos, esta função verifica se o documento XML apresenta um formato
correto e após isso vai começar a efetuar o parsing de caga group;
\item ParseGroup percorre um group do XML e extrai a informação correspondente. Se o grupo em questão
contiver outros groupos dentro de si, essa informação também será processada. 
Caso a informação a ler seja uma transformação a função parseAttributes é invocada;
\item ParseModel processa a informação correspondente ao ficheiro da figura primitiva(vértices) e a cor;
\item ParseAttributes que processa a informação sempre que é encontrada um transformação (traslate, rotate ou scale) criando
um vértice e adicionando-o ao Group que está a ser elaborada pelo
parseGroup.
\section{Exemplos de Execução}
\subsection{Figuras primitivas}
\subsection{Figuras primitivas com diferentes escalas}
\subsection{Sistema solar estático}

\end{itemize}
\section{Conclusão}
\end{document}